\documentclass[12pt]{article}
\usepackage[english]{babel}
\usepackage[utf8x]{inputenc}
\usepackage{amsmath}
\usepackage{graphicx}
\usepackage{etoolbox}
\usepackage{changepage}
\usepackage{titlesec}
\usepackage[parfill]{parskip}
\usepackage[margin=1in]{geometry}
\usepackage{times}
\usepackage{float}
\usepackage[numbers,super]{natbib}
\usepackage{lipsum} % Package to generate dummy text throughout this template. Remove for real use.

\titleformat*{\section}{\normalsize\bfseries} % Makes section titles 12 pt font


%----------------------------------------------------------------------------------------
%  TITLE SECTION
%----------------------------------------------------------------------------------------
\title{\large \textbf{How Social Media Algorithms Manipulate Users and Push Harmful Content}} % using \large makes the title approximately 14 pt.
% Author info isn't included for the Annual Conference but some regional conferences might request it.
\author{Marlen Vining}
%\author{\normalsize Author Name\\
%\normalsize email@example.com\\
%\normalsize Name of Your Department\\\
%\normalsize Your Institution Name}

\makeatletter % This gets the margins for the title set.
\makeatother

%----------------------------------------------------------------------------------------

\begin{document}
\raggedright
\maketitle
\thispagestyle{empty}
\pagestyle{empty}

%----------------------------------------------------------------------------------------
%  PAPER CONTENTS
%----------------------------------------------------------------------------------------
\section*{Abstract}
Social media platforms primarily generate revenue from the collection of user data and targeted advertising. This is why it is in their best interest to keep users engaged as long as possible. To do this, algorithms are designed to serve content to users that will keep them using the platform. Many of these algorithms purposefully serve content to users that is designed to trigger strong emotional responses. This is done to extend the time users spend on the platform and create a dependence. This often has a direct negative impact on the mental health and well being of users. The goal of this paper is to understand the tactics used by these platforms and uncover the harm it causes.
%------------------------------------------------

\section*{How Social Media Algorithms Work}
Social media platforms use algorithms to show users content that will keep them engaged with the platform. This is done by analyzing the habits of users to create a profile of what content is most effective at keeping them on the site."The reason why social media platforms use algorithms is to more organically filter through the large amount of content that is available on each platform. Algorithms do the work of delivering content that is potentially more “interesting” for a user to the detriment of posts which are deemed irrelevant or low-quality - either in general, or to a specific user."\cite{algorithms2021} These algorithms are typically designed to prioritize engagement over truth and what is healthy for a user to see.
%------------------------------------------------

\section*{Why Algorithms Push Harmful Content}
Social media sites have a financial incentive to keep users on the platform as long as possible. These sites are also incentivized to push content that increases the effectiveness of advertising on the platform. "Algorithms  designed  to  enhance  user  engagement  have  significantly  shaped  how content  is  prioritized, recommended, and disseminated. While these algorithms aim  to  create a  personalized experience, they often operate  without  sufficient safeguards against the spread of  harmful content,  including racial  and ethnic misinformation."\cite{Lucas2024} Content that produces a strong emotional response in the user, as well as content that builds a dependence on the platform keeps users engaged longer and creates a pattern of repeated use. Because of this, algorithms are often designed to push these kinds of contents. While this may be an unintentional result, it can still cause real harm. 

%------------------------------------------------

\section*{The Impact On Users}


%------------------------------------------------

\section*{Conclusion}


%----------------------------------------------------------------------------------------
%  REFERENCE LIST
%----------------------------------------------------------------------------------------
\vspace{4\baselineskip}\vspace{-\parskip} % Creaters proper 4 blank line spacing.
\footnotesize % Makes bibliography 10 pt font.
\bibliographystyle{acm} %Can use a different style as long as it is one which uses numbered references in the text.
\bibliography{paper}

%----------------------------------------------------------------------------------------



\end{document}

